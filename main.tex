\documentclass[a4paper]{article}


\input{./preambles/preamble}
\input{./preambles/unicode}
\setdefaultlanguage{english}
\setotherlanguages{german,french,greek,latin}

\input{./preambles/math-single}
\input{./preambles/math-brac}
\input{./preambles/math-thm}
\input{./preambles/phys-chem}

\newcommand{\RomaN}[1]{%
  \textup{\uppercase\expandafter{\romannumeral#1}}%
}


\usepackage[
	style = alphabetic,
	backend = biber,
	hyperref = true,
	backref = true,
	sorting = nyt,
	sortlocale = auto,
]{biblatex}
\addbibresource{gravitation.bib}
\addbibresource{landafshitz.bib}

\title{Publications in Gravitational Physics}
\author{Yi-Fan Wang}


\begin{document}
\maketitle

%\begin{abstract}
%Your abstract.
%\end{abstract}

%\tableofcontents

\section{General Relativity}

\subsection{Formalism}

The surface term was rediscovered (?) by \cite{Gibbons1977,York1972} and thus
(?) named as the Gibbons--Hawking--York term.

The Arnowitt--Deser--Misner formalism was first published in \cite{Arnowitt1959}
and was reviewed in \cite{Arnowitt2008} (reprint, original needed), editorial
note: \cite{Pullin2008}.

\subsection{Exact solution}

\section{Gauge Theories of Gravitation}

Friedrich Hehl on Poincar\'e gauge theory: \cite{Hehl1976}, status unclear.

Friedrich Hehl on Metric-affine theory: \cite{Hehl1995}, status unclear.

\section{Cosmology}

\subsection{Inflation}

Dong-Gang Wang has suggested \cite{Baumann2011}

\section{Text books}

\subsection{Course of theoretical Physics}
Курс теоретической физики; Lehrbuch der theoretischen Physik, 理论物理学教程

\paragraph{Mechanics}
\cite{Landau_1982a,Landau_1997a,Landau_2010a,Landau_2015a}

\paragraph{The Classical Theory of Fields}
\cite{Landau_1987a,Landau_1992a,Landau_2012a,Landau_2016a}

\paragraph{Quantum Mechanics: Non-Relativistic Theory}

\paragraph{Quantum Electrodynamics}

\paragraph{Statistical Physics}

\paragraph{Fluid Mechanics}

\paragraph{Theory of Elasticity}

\paragraph{Electrodynamics of Continuous Media}

\paragraph{Statistical Physics, Part 2: Theory of the Condensed State}

\paragraph{Physical Kinetics}

Rest of the current English volumes are
\cite{Landau_1986a,Landau_1980a,Landau_1980b,Lifshitz_1980a,%
Landau_1977a,Berestetskii_1982a,Lifshitz_1981a,Landau_1987a,%
Landau_1984a} (cleaning and linking to Russian versions needed).

Rest of the current German volumes are
\cite{Landau_2012,Landau1991,Landau_1987,Landau_1991a,%
Landau_1991,Landau_1990,Landau1992,Landau1990} (cleaning and linking to Russian
versions needed).

\subsection{A Shorter Course of theoretical Physics}

\cite{Landau_1974a,Landau_1972a} (cleaning and linking to Russian versions needed)

% Let's print the overall heading of the bibliography first:
%\printbibheading

\defbibcheck{undisting}{
	\iffieldequalstr{series}{Курс теоретической физики}{\skipentry}{}
	\iffieldequalstr{series}{Lehrbuch der theoretischen Physik}{\skipentry}{}
	\iffieldequalstr{series}{理论物理学教程}{\skipentry}{}
	\iffieldequalstr{series}{Course of Theoretical Physics}{\skipentry}{}
}
\printbibliography[check = undisting]

\defbibcheck{landafshitz_ru}{
	\iffieldequalstr{series}{Курс теоретической физики}{}{\skipentry}
}
\printbibliography[
	title = {Landau \& Lifshitz, Russian},
	check = landafshitz_ru,
	]

\defbibcheck{landafshitz_de}{
	\iffieldequalstr{series}{Lehrbuch der theoretischen Physik}{}{\skipentry}
}
\printbibliography[
	title = {Landau \& Lifshitz, German},
	check = landafshitz_de,
	]

\defbibcheck{landafshitz_zh_cn}{
	\iffieldequalstr{series}{理论物理学教程}{}{\skipentry}
}
\printbibliography[
	title = {Landau \& Lifshitz, Simplified Chinese},
	check = landafshitz_zh_cn,
	]

\defbibcheck{landafshitz_en}{
	\iffieldequalstr{series}{Course of Theoretical Physics}{}{\skipentry}
}
\printbibliography[
	title = {Landau \& Lifshitz, English},
	check = landafshitz_en,
	]

\end{document}